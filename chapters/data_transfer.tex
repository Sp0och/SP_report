\chapter*{Data Transfer}\label{ch:data_transfer}

\section*{Multiplayer Connection}\label{sec:multiplayer}

As mentioned above the approach to fullfil the data sharing requirement was to use an Unreal Engine multiplayer Connection.

There are different possibilites of online subsystems to use. As the plan is to use the handheld device on the construction site a local network connection was the choice for this project. Unreal Engine offers different types of multiplayer connections. Mainly two options could be relevant:
\begin{itemize}
    \item Client - Server model
    \item Dedicated Server model
\end{itemize}

The first option allows both the server and the client to actively interact with the game. This is useful in case we want to perform any UE action on the excavator. In this case the excavator's UE instance would be the playing the part of the server and the handheld device would be the client. 

The second option lets an Unreal Engine instance function as a real server without the possibility of interacting actively with the game. With this option the Unreal Engine instance on the excavator would serve as the dedicated server. In general this setup provides a more stable multiplayer connection however the possibility of interacting actively with the running UE instance would no longer be possible on the excavator.

In this project I used the Client - Server approach in case the necessity of input based interaction with the excavator UE instance arises in the future.

\subsection*{Connection Challenge}\label{subsec:key}

Normally when connecting two Unreal Engine games through a multiplayer connection it is the same game simply played from different devices. In this case however we have one UE instance running on the excavator using Linux both for development (ROS) and for the deployment (ROS Subscriber). The other game runs on Android and is an AR game. So in this setting we have a connection between two games which have fundamentally different components as well as different platforms that they run on. 

It turns out that on the lowest level a successfull UE multiplayer connection requires identical checksums in the two systems. The checksum is a number generated from Unreal Engine which depends on the project name and certain components in the game level amongst other factors.

The AR components

\section*{State Sharing}\label[type]{sec:state_sharing}
To establish all data transfer connections between the two devices would have been too wide of a scope for this project which is why I focused on the transmission of the excavator's current state. In general if the transmission of any information succeeds using this setup then all further information can be packaged into game components accordingly and shared in the same fashion.

As seen in \cref{fig:ex_setup} the excavator's state is fed to the Unreal Engine model directly through a ROS state subscriber node which is embedded in the game.

To solve the data transmission problem for the handheld device the idea was to intercept this connection. The state input is thus not only sent to the excavator model but also into a state storage game component. This game component would then be updated continuously and replicated in the handheld UE instance with the new values. Simultaneously if an excavator model is referenced by this state storage actor the state is fed into the model at each game tick to also update the handheld excavator model's state.
