\chapter{Introduction}
\label{ch:introduction}


When operating an excavator in the conventional fashion two requirements are imperative: \begin{itemize}
    \item The view onto the construction site
    \item The handles necessary to give inputs to the machine
\end{itemize}
In the case of an autonomous excavator the low level actions to perform are determined by the machine itself, given a high level action input. So in this case the second requirement becomes the possibility to give such a high level input to the system.

With the desired handheld remote control setup the necessity of sharing information between the device and the machine still persists. The visual requirements change however. From the handheld camera we now receive the required view of the surroundings on the construction site but in order to successfully share a geometric location with the excavator the colocalization problem has to be solved for the two entities   . Having an AR remote control integrated in a handheld device also provides the possibility of introducing further useful features such as displaying a preview of an action of choice. 

In this project I attempted to overcome the key challenges constituting the requirements mentioned above. 

The plan was to solve the data transfer requirement utilizing the already implemented Unreal Engine setup of the autonomous excavator\footnote{HEAP - The autonomous walking excavator\citep*{heap}} from the Robotic Systems Laboratory. To account for the colocalization problem an approach using Microsoft's Azure Spatial Anchors was used.