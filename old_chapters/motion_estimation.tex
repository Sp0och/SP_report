\chapter{Motion Estimation}\label{ch:motion_estimation}

Instead of applying a computationally expensive iterative procedure like ICP\footnote{Iterative Closest Point \citep{Pomerleau}} I could make use of my previously found correspondences and apply the closed form solution\citep{closed_form} to the alignment problem.

So each iteration a rotation matrix \textbf{R} and a translation vector \textbf{t} are estimated.

\section{Pose update}{
    In order to track the estimated pose of the sensor a current pose is considered in homogeneous coordinates and is updated using each iterations respective rotational and translational change.

    \begin{center}
        $\text{New Pose} = 
        \underbrace{\begin{bmatrix}
            R_{11o} & R_{12o} & R_{13o} & t_1o\\
            R_{21o} & R_{22o} & R_{23o} & t_2o\\
            R_{31o} & R_{32o} & R_{33o} & t_3o\\
            0&0&0&1
        \end{bmatrix}}_{\text{Old Pose}} * 
        \underbrace{\begin{bmatrix}
            R_{11i} & R_{12i} & R_{13i} & t_1i\\
            R_{21i} & R_{22i} & R_{23i} & t_2i\\
            R_{31i} & R_{32i} & R_{33i} & t_3i\\
            0&0&0&1
        \end{bmatrix}}_{\text{Pose Update}}
        $
    \end{center}

    This pose represents the transformation from the static world frame to the dynamic sensor frame and determines the quality of this methods motion estimation procedure.
    
    For the evaluation of my method and the comparison of the individual building blocks used in my method I considered the transformation as such, the path drawn when following the pose in the world frame as well as the mapping quality when considering multiple subsequent point clouds. 
}






